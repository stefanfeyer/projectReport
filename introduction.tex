\chapter{Introduction}

\section{Introduction}
\begin{figure}
	\centering
	\includegraphics[width=1.0\textwidth]{img/perspectives.PNG}
	\caption{The four perspectives the system is capable of. Human figure taken from thenounproject.com, accessed: 19.06.2020. Icon: created by Ghan Khoon Lay from Noun Project.}
	\label{fig:perspectives}
\end{figure}
Motor learning is necessary for various activities, like the ergonomic conduction of working routines, sports, arts, or dancing. Training such movements usually include a teacher that performs the movements, the learner (further called \textbf{student}) can mimic. However, a teacher is not always accessible because of the lack of availability, economic reasons, or time. In this case, technical solutions exist to step into this role. For example, YouTube\footnote{\href{https://www.youtube.com/}{youtube.com}} and other video platform turned into a great source of learning videos.\\
Research proved that Mixed Reality (MR) could also be a valuable technology to teach students. MR system can provide teachers no only in two dimensions (2D) but in three dimensions (3D), which provides a better grasping of the movement in question. Also, this system can be interactive and provide feedback on the performance of the student. On this assumptions, researcher developed training systems for arts \cite{Han2016, Komura2006, Chua}, dance \cite{Yan2015, Chan2010, Hachimura2004}, sports \cite{Covaci2014, Kojima2014} and rehabilitation \cite{Chinthammit2014, Tang2015, Rajanna2015}, in the ego-centric \cite{Yang2002, Katzakis2017, Scavo2015}, exo-centric \cite{Han2017, Velloso2013, Lieberman2007} or both \cite{Sousa2016, Hoang2016, Sodhi2012} visual perspectives. These systems showed that the choice of the visual perspective on a guidance visualisation (further also called \textbf{teacher}) could potentially influence motor learning.\\
On this basis, this work aims to answer the following research question:
\begin{tcolorbox}[colback=red!30!white]
	Does the visual perspective on a virtual guidance visualisation have an influence on motor learning in MR environments?
\end{tcolorbox}
The preceding seminar thesis showed that investigation on this topic is valuable because knowledge about how the visual perspective on a guidance visualisation influence motor learning is limited.\\
Furthermore, the seminar thesis investigated on several aspects of motor learning and revealed requirements a motor learning system must fulfil to train motions effectively. Additionally the scope was set:
\begin{itemize}
	\item serial movements
	\item cognitive stage
	\item closed skills
	\item single error measures
	\item virtual reality (VR)
\end{itemize}
For this scope, the requirements are as follows.
\begin{itemize}
	\item[R1] provide at least four different perspectives
	\item[R2] provide different view points on the guidance visualisations
	\item[R3] avatar guidance visualisation
	\item[R4] high realism degree, person shaped avatar
	\item[R5] no additional feedback
	\item[R6] scaling of the teachers height to the students height
	\item[R7] provide measures
	\item[R8] provide full body motion tracking
\end{itemize}
The four perspectives will be the ego-centric perspective, the exo-centric perspective, as well the combinations ego \& exo centric perspective and the augmented exo-centric perspective compare \ref{fig:perspectives}.

In the \textbf{ego-centric} visual perspective the teacher stands inside the student, in the \textbf{exo-centric} visual perspective outside the student. In the \textbf{ego \& exo centric} perspective, the teacher stands as well as inside and outside the student. In the \textbf{augmented exo-centric} visual perceptive the teacher stands inside a virtual copy of the student.\\
The task will relate to how to handle physical load for the following reasons: it might help to address critical health issues by allowing the guidance of the correct ergonomic conduct. Furthermore, there is view research on guidance on how to handle physical load.\\
This work starts with an evaluation of the hardware and software, followed by the implementation of the system and finally, the perspectives. In the end, an outlook is provided.
