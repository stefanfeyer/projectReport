\chapter{Introduction}

\section{Introduction}

Motor learning is necessary for various activities, like the ergonomic conduction of working routines, sports, arts, or dancing. Training such movements usually include a teacher that performs the movements, the learner (further called \textbf{student}) can mimic. However, a teacher is not always accessible because of the lack of availability, economic reasons, or time. In this case, technical solutions exist to step into this role. For example, YouTube\footnote{\href{https://www.youtube.com/}{youtube.com}} and other video platform turned into a great source of learning videos.\\
Research proved that Mixed Reality (MR) could also be a valuable technology to teach students. MR systems can provide teachers no only in two dimensions (2D) but in three dimensions (3D), which provides a better grasping of the movement in question. Also, MR systems can be interactive and provide feedback on the performance of the student. On this assumptions, researcher developed training systems for arts \cite{Han2016, Komura2006, Chua}, dance \cite{Yan2015, Chan2010, Hachimura2004}, sports \cite{Covaci2014, Kojima2014} and rehabilitation \cite{Chinthammit2014, Tang2015, Rajanna2015}, in the ego-centric \cite{Yang2002, Katzakis2017, Scavo2015}, exo-centric \cite{Han2017, Velloso2013, Lieberman2007} or both \cite{Sousa2016, Hoang2016, Sodhi2012} visual perspectives. These systems showed that the choice of the visual perspective on a guidance visualisation (further also called \textbf{teacher}) could potentially influence motor learning.\\
However, how the visual perspectives on virtual guidance visualisation influences motor learning is widely left out in these works. The fact that the visual perspective on virtual guidance visualisations potentially influences motor learning in combination with the limited knowledge of how the visual perspective influences motor learning shows the necessity of investigations.\\
On this basis, this work aims to answer the following research question:
\begin{tcolorbox}[colback=red!30!white]
	Does the visual perspective on a virtual guidance visualisation have an influence on motor learning in MR environments?
\end{tcolorbox}
The preceding seminar thesis investigated on several aspects of motor learning and revealed requirements a motor learning system must fulfil to train motions effectively. In the following, a brief overview of scope and requirements is provided. For a more detailed elaboration, see seminar thesis. The scope was set to:
\begin{itemize}
	\item Serial movements: a sequence of discrete movements, best suited for a study and widely used by other researchers \cite{Anderson2013, Chan2010, Hoang2016}.
	\item Cognitive stage: highest increase of proficiency, therefore best suited for a study.
	\item Closed skills: investigation in a controlled environment.
	\item Single error measures: proven suitable by other researchers to generate data to evaluate motor learning \cite{Sodhi2012, Tang2015, Komura2006}.
	\item Virtual reality (VR): AR provides a limited field of view, VR more suitable for the study in question.
\end{itemize}
For this scope, the requirements are as follows.
\begin{itemize}
	\item[R1] Provide at least four different perspectives: ego-centric, exo-centric, ego \& exo-centric and augmented exo-centric seem most promising to deliver insights to answer the research questions. Further elaboration in next section \textit{Perspectives}.
	\item[R2] Provide different viewpoints on the guidance visualisations: mirror effect and occlusion of the guidance visualisation makes it necessary to provide different viewpoints on the guidance visualisation \cite{Chua}.
	\item[R3] Avatar guidance visualisation: a person-shaped avatar will be used as virtual guidance visualisation like used in \cite{Chua, Komura2006, Han2017}.
	\item[R4] High realism degree, person-shaped avatar. Weber \cite{Weber2018} showed that high realism guidance visualisations are more suited for virtual guidance visualisations.
	\item[R5] No additional feedback: evaluation is focused on the perspective; feedback will not be part of the evaluation.
	\item[R6] Scaling of the teachers height to the student's height: in case of teacher an students have different body sizes, especially in the ego-centric perspective, scaling of the teacher is mandatory \cite{Hoang2016}.
	\item[R7] Provide measures: the system must be capable of producing reliable data for the evaluation.
	\item[R8] Provide full-body motion tracking: real-time tracking of the student.
\end{itemize}

\section{Perspectives}
In a scenario with one teacher and one student, five different visual perspective are possible: ego-centric, exo-centric, ego \& exo-centric, augmented exo-centric and ego \& augmented exo-centric, compare \ref{fig:perspectives}. To reduce complexity, the focus is set to the fist four perspectives. These four visual perspectives seem most promising to deliver insights to answer the research question.\\
In the \textbf{ego-centric} visual perspective the teacher stands inside the student, in the \textbf{exo-centric} visual perspective outside the student. In the \textbf{ego \& exo centric} perspective, the teacher stands as well as inside and outside the student. In the \textbf{augmented exo-centric} visual perceptive the teacher stands inside a virtual copy of the student.\\
\begin{figure}
	\centering
	\includegraphics[width=1.0\textwidth]{img/perspectives.PNG}
	\caption{The four perspectives the system is capable of. Human figure taken from thenounproject.com, accessed: 19.06.2020. Icon: created by Ghan Khoon Lay from Noun Project.}
	\label{fig:perspectives}
\end{figure}

\section{Task}
The task will relate to how to handle physical load for the following reasons: it might help to address critical health issues by allowing the guidance of the correct ergonomic conduct. Furthermore, there is view research on guidance on how to handle physical load.\\
Physical load hereby could be represented by a box. The virtual guidance visualisation shows a sequence of movements like lifting, turning, pushing the box on or onto a table. Examples can be found in the following video segment: \href{https://youtu.be/XR5dDxk40WM?t=228}{click}.

\section{Structure}
This masters project aims to implement a VR motor learning system that can be used in a study to evaluate the above-mentioned research question. This document describes the implementation of E(x\textbar g)o, which claims to be such a system. E(x\textbar g)o fulfils the requirements developed in the preceding seminar thesis. The development starts with an analysis of hard and software in chapter 2, which serves as a fundament for the implementation described in chapter 3. Chapter 3 points out the limitations of E(x\textbar g)o, too. In chapter 4, an outlook provides insight into the next steps.

