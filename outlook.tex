\chapter{Outlook}

\section{Upcoming Tasks}
E(x\textbar g)o is a system capable of training a student the handling of physical load in Virtual Reality in 5 different visual perspectives, including the perspective which probably not be used in the study. For an evaluation of these perspectives, now the measures need to be implemented, namely the ergonomic measurement and the precision measurement described in the next section. Comfort of study participants is important during a study. For this, special straps are planned to replace the OptiTrack suit. Furthermore, the box the student is handling needs to be replaced with more elaborated one.

\section{Evaluation}
The evaluation of E(x\textbar g)o can be conducted in two ways. The first is a comparable study, with the conditions ego-centric, exo-centric, ego \& exocentric and augmented exo-centric. Here the participants have the task to handle a box ergonomically. Measures for ergonomic handling the box give insights on how well the participant could follow the instructions of the teacher. These measurements consist of spine twist, spine bend and foot placements like described by Muckell et al.\cite{Muckell2017}. Additionally, a precision measurement can be applied. The precision is defined as the Euclidean distance between the students box and the teachers box. The study is a within subject design with counterbalancing resulting in at least 16, better 32 or 48 participants. Because of the current corona situation, an only initial study with less participants could be conducted.\\
The other possibility to evaluate E(x\textbar g)o is to make a evaluation of the design decisions. In this case, the tethering distance in the ego-centric could be refined, the size and weight of the box, as well as the formation of the teachers in exo-centric perspectives.