\chapter{Implementation}
The preceding chapter describes the software and hardware which serves as the fundament the implementation builds on. To create a system to teach motor learning from multiple perspectives several steps have to be made: the Vive Tracker matching, VRIK calibration, the rendering of the student, teacher, table and box. With these steps completed the movements of a teacher can be recorded. The recorded teacher then needs to be resized to the students height. With all this set up, the visual perspectives themself need to be developed.\\
This chapter describes in detail the implementation of theses steps. The last section shows a comparison of the perspectives with their mechanics utilised to achieve the visual perspective in question, as well as their limitations.

\subsection{Vive Tracker Matching}
\begin{sidewaysfigure}
	\centering
	\includegraphics[width=1.0\textwidth]{img/viveTrackerMatching.png}
	\caption{Vive Tracker matching. Left: reference holding all Objects to be transformed by a tracker, middle: device ID list, right script which reads the hardware IDs and sets the device IDs.}
	\label{fig:viveTrackerMatching}
\end{sidewaysfigure}
In Unity, a Vive Trackers position and orientation can be assigned to a GameObject with the script \textit{SteamVRTrackedObject} compare figure \ref{fig:viveTrackerMatching} middle. In this script a device ID is used to be associated with a physical Vive Tracker. Unfortunately, every time SteamVR driver is started, the device IDs change. For example device ID 1 was associated with the Vive Tracker placed on the left feet of the user. On the next start-up, device ID 1 can be associated with Lighthouse base station. This leads to no study-safe setup and must be corrected. To solve this issue, first all physical Vive Tracker needs a label. The label is a sticker with the numbers B1-B12. After that, device hardware IDs of the used devices must be read out. The device IDs per tracker label in use are:
\begin{itemize}
	\item[B1:] LHR-67E402D1 Hip
	\item[B3:] LHR-32C38603 RFoot
	\item[B5:] LHR-4E4C94A4 LFoot
	\item[B6:] LHR-B925C963 Table
	\item[B8:] LHR-89131158 BOX
	\item[B9:] LHR-31D0CDF2 LHand
	\item[B10:] LHR-CAC69A3C RHand
\end{itemize}
Secondly, the hardware id must be matched to specific devices. Then, at the start-up of the system, an algorithm reads out all hardware IDs and assign it to the correct \text{GameObject} by associating the correct device ID. Since this is a common issue across many projects, this code illustrates the solution:

\begin{lstlisting}[language=JAVA, frame=single]
for(uint i = 0; i < 16; i++){
 ETrackedPropertyError error = new ETrackedPropertyError();
 StringBuilder sb = new StringBuilder();

 OpenVR.System.GetStringTrackedDeviceProperty(i,
  ETrackedDeviceProperty.Prop_SerialNumber_String, 
  sb, OpenVR.k_unMaxPropertyStringSize, ref error);

 var serial = sb.ToString();

 switch (serial)
 {
  case "LHR-67E402D1":
   // "Found device with ID LHR-67E402D1 (Hip).
   // Assinging Hip with device index: " + i
   studentHip.GetComponent<SteamVR_TrackedObject>()
    .SetDeviceIndex((int)i);
  break;
  case "LHR-32C38603":
   // "Found device with ID LHR-32C38603 (RFoot).
   // Assinging RFoot with device index: " + i
   studentRF.GetComponent<SteamVR_TrackedObject>()
    .SetDeviceIndex((int)i);
  break;
...

}
\end{lstlisting}

\subsection{VRIK Calibration}
%yxc image of person wearing the trackers, point out the trackers
VRIK provides natively to track a person in Virtual Reality. But the setup is rather complex and needs several calibrations for the head, hands and feet. The \textit{VRIK Calibration Controller} brings all calibrated parts together. The following sections describe the calibration steps needed to generate natural looking representations of the student and teacher. This process is extensively time consuming, because all values need to be determined by experimentation. The values for calibration can be reused in similar setups for other projects.
\begin{figure}
	\centering
	\includegraphics[width=1.0\textwidth]{img/student_calibration.png}
	\caption{All elements and their relations to calibrate one student. Other projects with a similar setup can reuse the values of shifting and rotating the markers}
	\label{fig:student_calibration}
\end{figure}
\subsubsection{Head}
VRIK solves the limbs based on the tracking points mentioned above. The head reference is a child of the \textit{camera} provided by the SteamVR asset. It holds position and rotation and is the most important input for the solver. The VRIK solver takes this position as absolute and solves everything else based on this position. The position but not rotation overrides even the parent transforms. This fact influences the further handling in Unity massively: to transform a VRIK, the GameObject can not be translated into a specific position. This translation can only happen by transforming the position of the references of the Vive Tracker itself. Because of this, a student avatar never holds the original head transform of the head. The avatar holds an GameObject that copies the transform of the head transform.
\subsubsection{Hands}
Left and right hand position need to be adjusted because their root lies inside the body in the middle of the joint. To transform the References accordingly, they are a child to the tracker reference. From this root, the references are shifted 4 cm towards X and 3 cm towards Y. Additionally, the rotation needs to be adjusted by 90$^\circ$ in Y and Z. With this, the tracking point is located in the middle on the back of the hand. The values used for hand calibration are depicted in figure \ref{fig:student_calibration}
\subsubsection{Feet}
Left and right feet need adjustments in terms of position and rotation, too. The Vive Tracker is located at the top of the feet, which is fine with the root of the VRIK reference. However, the slope of the trackers and forward direction must be adjusted. The left reference needs a tilt of -25$^\circ$, the right reference by +25$^\circ$. For the correct direction, the left reference needs a rotation by 30$^\circ$ on Z-axis, the right by 60$^\circ$.\\
The hip does not need further shifts. The values used for feet calibration are depicted in figure \ref{fig:student_calibration}

\subsubsection{VRIK Callibration Controler}
With the correct position of the references, the references can now be assigned to a VRIK instance. This works with a VRIK calibration controller shown in figure \ref{fig:calibrationController}.
\begin{figure}
	\centering
	\includegraphics[width=1.0\textwidth]{img/calibration_controller.PNG}
	\caption{VRIK Calibration Controller. Sets the references necessary to calibrate the student.}
	\label{fig:calibrationController}
\end{figure}
The calibration controller takes the references and assigns it to the VRIK instance. In the same step, the humanoid character is sized to the references.

\subsection{Student Rendering}
\begin{sidewaysfigure}
	\centering
	\includegraphics[width=1.0\textwidth]{img/student_rendering_pipeline.png}
	\caption{Top: data flow to generate the visual representation of a student. Bottom: Example of data flow from the Tracker placed on the hip of a student until the visual representation.}
	\label{fig:student_rendering_pipeline}
\end{sidewaysfigure}
The rendering of the student's humanoid character is based on the live tracking data of the Vive Trackers. As seen above, the Trackers get associated to a GameObject, e.g. hip. Relative to this GameObject, a reference is created. this reference is passed to the calibration controller. The calibration controller sets the corresponding references in the VRIK solver. So far nothing new.\\
Now the VRIK takes a humanoid character and binds its muscles. then the solving begins, and the algorithm animates the associated humanoid character. This character itself consists of a hierarchy of bones and muscles. These bones and muscles are overlaid by a mesh renderer which produces the desired output, compare figure \ref{fig:teacher_rendering_pipeline}.

\subsection{Teacher Rendering}
\begin{sidewaysfigure}
	\centering
	\includegraphics[width=1.0\textwidth]{img/teacher_rendering_pipeline.png}
	\caption{Top: data flow to generate the visual representation of a teacher. Bottom: Example of animating the hip of a teacher.}
	\label{fig:teacher_rendering_pipeline}
\end{sidewaysfigure}
The naive approach of rendering the teacher is to record a teacher and then just play the animation. However, recording a humanoid animation is not natively supported by Unity and would require a third-party tool to do so. This leads to unwanted differences in the representation of the teacher. To ensure, that teacher and student produce comparable renderings; the same technique has to be utilised. To accomplish this requirement, the movements are recorded by recording the student's tracker position and then pass this as animations to a similarly shaped hierarchy of \textit{GameObjects}, that in turn are passed to the VRIK solver. This is why the rendering pipeline of the teacher differs from the students rendering pipeline.\\
The process for the teacher is as follows, compare figure \ref{fig:teacher_rendering_pipeline}:
For the teacher, a reference hierarchy needs to be created. The calibration above applies, but the head must be extracted from the camera and transformed into an own \textit{GameObject} with the heads reference turn by 90$^\circ$ on the Y and Z axis. On the parent \textit{TeacherReference} an \textit{AnimatorController} is attached. This \textit{AnimatorController} plays the pre-recorded animation and transforms the references of the children accordingly. Now, these children can be used like in the students rendering pipeline and passed to the \textit{CalibrationController}. Afterwards, the rendering pipeline is as seen in the student's pipeline.

\subsection{Assets Rendering}
\begin{figure}
	\centering
	\includegraphics[width=1.0\textwidth]{img/assets_rendering_pipeline.png}
	\caption{Data flow to generate the visual representation of the box and the table.}
	\label{fig:asset_rendering_pipeline}
\end{figure}
Finally, the objects the student and the teacher are interacting, namely the table and the box, need to be rendered. Rendering of these object is less complex. The student's objects get attached by a Vive Tracker. The position and rotation of the Tracker are bound to a cube, which is scaled to the size of the box or the table. The teacher's assets are also cubes scaled to the size of the cube and the table. But the position and rotation are set by the animator controller, compare figure \ref{fig:asset_rendering_pipeline}
\subsection{Recording of Movements}
As mentioned above, the recording of humanoid movements is not trivial. The solution is to create a \textit{GameObject} with a similar hierarchy, compare figure \ref{fig:record_animation}. This \textit{GameObject} is called \textit{Recorder}. It is similarly structured like the student. The references are holding a script which sets the transforms them into the transforms of the student. Additionally, the head reference, the box reference and the table reference. The transforms of these references are also copied from the real elements. This is necessary to have one single animation that controls the teacher. Otherwise, time shifts between animations could be possible. Now the complete \textit{Recorder} hierarchy can be recorded by the \textit{Unity Recorder}. The outcome is an animation that produces the references necessary to render the teacher like described above.
\begin{figure}
	\centering
	\includegraphics[width=1.0\textwidth]{img/record-animate.png}
	\caption{Left: setup of the recorder, GameObject that holds all points to record. Right: animated GameObject is used to steer the teacher's dummy.}
	\label{fig:record_animation}
\end{figure}

\subsection{Resize}
The teacher and the student are most likely to have different body sizes. Especially in the ego-centric perspective, the height of the teacher must match the height of the student. However, the naive approach of just resizing the teacher's humanoid character to the size of the student leads again not to a satisfying result. This is because the head position is absolute and not changeable for the VRIK solver. Resizing the teacher leads to floating or swaged representations. The solution for this is to resize the parent \textit{GameObject} of the teacher's references, and then pass the already scaled references to the calibration controller. However, this leads to another issue: the size of the student can only be assessed after the calibration controller did his work, assigned the references to the VRIK solver and sized the representation. To overcome this issue, a 2-step body size calibration is necessary. First, the student's calibration controller performs the calibration of the student. Then the student's height is measured and compared to the teachers size and stored in $\Delta y$. $\Delta y$ describes the percentage of the size difference between teacher and student. Afterwards, the teacher's references root is scaled by $\Delta y$ and gets reassigned to the calibration controller, which itself performs the task again. With this, the teacher has the same size as the student. But now the box and table are also resized because they are in the same parent \textit{GameObject}. However, with the knowledge about $\Delta y$ the box and the table can be upscaled again by $\frac{1}{\Delta y}$ to have the original size.\\
The length of arms torso and feet are resized by VRIK Calibration Controller, which means a simple superordinate scaling is a valid resize method.
\begin{figure}
	\centering
	\includegraphics[width=1.0\textwidth]{img/resize.png}
	\caption{Calculating the height difference of $\Delta y$ between the teacher and the student.}
	\label{fig:resize}
\end{figure}


\section{Visual Perspectives}
With the previous implementations done, E(x\textbar g)o is capable to show a teacher and a student. This section describes the implementation of the visual perspective in which the student can learn movements. Namely, ego-centric, exo-centric, ego \& exo-centric and augmented exo-centric, compare figure \ref{fig:perspectives}\footnote{Remark: E(x\textbar g)o is also capable of the fifth possible visual perspective ego \& augmented exo-centric. Because the focus is set to only the first four visual perspectives, the description of the fifth visual perspective is not part of this report.}.

\subsection{Ego-Centric}
\begin{figure}
	\centering
	\includegraphics[width=1.0\textwidth]{img/continuumShift.png}
	\caption{Continuum of perspectives according to Milgram \cite{Milgram1994}. The red arrow indicates the shift towards the exo-centric perspective.}
	\label{fig:continuumShift}
\end{figure}
The nature of an ego-centric perspective raises one big issue to tackle: the student has "to be in the teacher" at any point in time. If the teacher now wants to indicate a movement, meaning a translation of the own position in space, it is indicating this movement by moving away from the student. This leads inevitably to a non-ego-centric perspective. To solve this issue, a closer look at the definition of the visual perspectives can help, compare figure \ref{fig:continuumShift}: a continuum represents the visual perspectives. On one extreme, the ego-centric visual perspective is located, while on the other extreme, the exo-centric perspective is located. With moving from one extreme to the other, the tethering distance\footnote{The tethering distance is the distance between the eyes anchor point and the camera observing the character in question.} is changing. A tethering distance of 0 indicates a pure ego-centric perspective, while a greater tethering distance means shifting towards the exo-centric extreme. Following the nature of a continuum, a slightly larger tethering distance than 0 is still an ego-centric perspective. The task is now to choose a value for the tethering distance that it still is an ego-centric perspective, but at the same time let the student be able to interpret the teacher's indication of movement.\\
To determine a reasonable tethering distance, a small study was conducted with two persons\footnote{Due to COVID-19 a larger study was not possible.} In this study the tethering distance was increased continuously from 5 to 50 cm. The best result was yielded between 15 and 30 cm. Based on this empirical values, the tethering distance is set to 30cm. It still feels ego-centric, but the indication of the movement of the teacher is clear to recognise. However, a hard threshold would lead to non-fluid demonstrations of the teacher, which is hard to follow for the student. For this reason, the tethering is set to a frame of 15cm to 30cm with a variable, interpolated speed of the teacher's demonstration. This means, between 0cm and 15cm, the teacher's movement demonstrations are shown at normal speed. Between 15cm and 30cm, the speed linearly decreases and stops completely at a tethering distance of 30cm. This makes the teacher's movements easily to follow by the student with no interruptions.\\
In the following, the implementation of this mechanic is described. First, the teacher must be translated to the position of the student. Again, the naive approach of just shifting the teacher's avatar on the student's position does not fulfil the requirements, because the VRIK solver takes the position of the references as absolute. Though, the teacher's references parents zero holding the animator has to be transformed.
\begin{figure}
	\centering
	\includegraphics[width=1.0\textwidth]{img/rootTransform.png}
	\caption{Process of shifting the teacher into the student.}
	\label{fig:rootRansform}
\end{figure}

For this, the distance between the parent of the references zero of teacher and student is calculated, and then the teacher's references parent is shifted by this distance. This process is shown in detail in figure \ref{fig:rootRansform}: the teachers root must be shifted by $\vec{d}$, while $\vec{d}$ is determined by $\vec{a}$, $\vec{b}$ and $\vec{c}$. After shifting the teacher's root by $\vec{d}$, the animation ($\vec{c}$) of the teacher applies, and the teacher is placed directly in the student. Secondly, the animation speed is set by the distance between the student and the teacher. For this calculation, the hip references of both are taken into consideration. Is the difference between these two points below 0.15, the animation speed of the teacher is 1. If the distance is greater than 0.3, the animation speed is 0. Between 0.15 and 0.3 the animation is a linear interpolation. The following pseudocode achieves this:

\begin{lstlisting}[language=JAVA, frame=single]
float stopDistance = 0.3f;
float fullSpeedDistance = 0.15f;

void Update()
 {
  Vector3 deltaStudentTeacher =
  studentHip.transform.position -
  (teacherHip.transform.position -
  teacherZero.transform.position);

 if(deltaStudentTeacher.magnitude > stopDistance){
  setAnimationSpeedOfAllTeachersToZero();
 }
 else
 {
 allTeacherAnimators.speed =
  Mathf.Min(1f, (stopDistance / fullSpeedDistance - 
  (deltaStudentTeacher.magnitude / fullSpeedDistance)));
 }   
}
\end{lstlisting}

\begin{figure}
	\centering
	\includegraphics[width=1.0\textwidth]{img/teachers_animation_speed_chart.png}
	\caption{Teachers animation speed, linear interpolated.}
	\label{fig:teachers_animation_speed_chart}
\end{figure}
%yxc how to achieve multiple representations?

\subsection{Exo-Centric}
The exo-centric visual perspective is less complex than the ego-centric visual perspective. The teacher is located outside of the student, and the student can choose their position to observe the teacher. However, in this condition, the student can turn away from the teacher by following the instructions of the teacher. To visualise this issue, follow this scenario (\href{https://www.youtube.com/watch?v=mRG22RBXKTM&feature=youtu.be}{demo video}): teacher and student standing side by side and looking into the same direction. The student stands left of the teacher. If the teacher turns left and the student also turns left, the student can no longer see the teacher and follow the instructions. However, even when the teacher is turning to the right, the student sees the teacher from the back, unable to see what the teacher is doing in front of the teacher's body. For this issue a solution must be found. Chua et al. \cite{Chua} decided to introduce multiple representations of the teacher and found the mechanic as an aid for motor learning. Additionally, ai a real-world scenarios, the same issue exists. However, in real-world scenarios where this issue is likely, it is also likely that in the room are mirrors, allowing to see areas that are occluded by the teacher's body.  VR provides the possibility to have multiple representations of the teacher around the student. With this argumentation, multiple representations of the teacher are introduced. Instead of having only one instance of the teacher, four instances on different positions around the student are present. To determine appropriate positions of the additional teachers, a small study was conducted with two people\footnote{Due to COVID-19 a larger study was not possible.} During interacting with the box, the teachers were shifted around the student. After 3 iterations, the teachers positions were set to:
\begin{itemize}
	\centering
	\item[teacher0:] $\vec{t_0} = (-0.2,0,1.5)$
	\item[teacher1:] $\vec{t_1} = (-1.3,0,-0.75)$
	\item[teacher2:] $\vec{t_2} = (-1.3,0,0.75)$
	\item[teacher3:] $\vec{t_3} = (-0.2,0,-1.5)$
\end{itemize}

Now the student has multiple angles on the teacher, which overcome the mirror issue\footnote{Mirror issue: a teacher is standing in front of the student forces the student to move the, e.g. left arm when the teacher moves the right arm. Compare seminar thesis for more details}, too.
\subsection{Ego-Centric \& Exo-Centric}
The combination of the ego-centric and exo-centric perspective leads to the third visual perspective. As the name indicates, the teacher can be seen from the ego-centric perspective and the exo-centric perspective simultaneously: the teacher stands outside and inside of the student's body. To achieve this, the two above visual perspectives are combined. Starting from the above described exo-centric visual perspective with its four teachers, a fifth teacher is introduced and placed inside of the student with the same mechanics as in the above described ego-centric perspective. Eventually, the animation speed of the ego-centric teacher must be applied to the exo-centric teachers. This happens by the script \textit{putTeacherIntoStudent.cs} which holds references to all animation controllers. The script synchronises all animations and changes the speed of the animation playback.
\subsection{Augmented Exo-Centric}
In the augmented exo-centric perspective, the student stands inside of the teacher. While the number of teachers remains at 4, the student now needs to have multiple representations. One at the real-world location, representing the ego-centric perspective, and four more in every teacher. The formation of the exo-centric perspective remains. Because no ego-centric perspective is present, the animation is not needed to have speed interpolation.

\section{Limitations: Perspectives and Mechanics}
The four visual perspectives utilise two mechanics for teaching a student movements. \textit{Multiple representations} are necessary to over come occlusion and the mirror issue. The \textit{speed} mechanic is used to indicate transitional movements in the ego-centric perspectives. \textit{Multiple representations} is used in all perspectives except for the ego-centric perspective, compare table \ref{fig:mechanic}. In the ego-centric perspective it is not necessary to show multiple representations. Therefore the missing of this mechanic only in this perspective is not study corrupting.\\
The \textit{speed} mechanic is applied in all perspective except the exo-centric perspective. The reason for this is, that the student has no visual indication on what point in space the movements should be performed. This makes it nearly impossible to the step into the threshold of 30 cm to trigger the teachers movement instruction.\\
Since the \textit{speed} mechanic is essential in the other conditions, the missing of this mechanic in only one condition could lead to evaluate the mechanic and not the movements of the student. A solution could be to provide a light spot on the ground to indicate the position. Another approach would be to replace the exo-centric visual perspective with the ego \& augmented exo-centric condition. The final solution to this issue will be part of the Masters thesis.
\begin{table}
	\centering
	\includegraphics[width=1.0\textwidth]{img/mechanics.png}
	\caption{Comparison of visual perspectives with applied mechanics}
	\label{fig:mechanics}
\end{table}
\ref{fig:mechanics}
